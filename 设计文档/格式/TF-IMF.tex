\documentclass{article}
\usepackage{amsmath}
\usepackage{CJKutf8}
\usepackage{geometry}
\geometry{a4paper,margin=2.5cm}

\title{TF-IDF算法数学公式说明}
\author{Web2.7项目组}
\date{\today}

\begin{document}
\begin{CJK}{UTF8}{gbsn}

\maketitle

\section{TF-IDF算法基本公式}

\subsection{词频(TF)}
词频(Term Frequency)表示词条在文档中出现的频率:

\[
TF(t,d) = \frac{n_{t,d}}{\sum_{k} n_{k,d}}
\]

其中:
\begin{itemize}
\item $n_{t,d}$ 表示词条t在文档d中出现的次数
\item $\sum_{k} n_{k,d}$ 表示文档d中所有词条的总数
\end{itemize}

\subsection{逆文档频率(IDF)}
逆文档频率(Inverse Document Frequency)用于衡量词条的重要性:

\[
IDF(t) = \log\frac{N}{df_t + 1}
\]

其中:
\begin{itemize}
\item $N$ 表示文档集合中的文档总数
\item $df_t$ 表示包含词条t的文档数量
\item 加1是为了避免分母为0
\end{itemize}

\subsection{TF-IDF得分}
TF-IDF得分是TF和IDF的乘积:

\[
TFIDF(t,d) = TF(t,d) \times IDF(t)
\]

\section{项目中的改进公式}

\subsection{标题权重加权}
对于标题中出现的词条,我们增加了额外的权重:

\[
Score_{title}(t) = TFIDF(t,d) \times (1 + \alpha_{title})
\]

其中 $\alpha_{title} = 0.8$ 是标题权重系数。

\subsection{位置权重}
对于在标题开头出现的词条,进一步增加权重:

\[
Score_{position}(t) = Score_{title}(t) \times (1 + \alpha_{position})
\]

其中 $\alpha_{position} = 0.5$ 是位置权重系数。

\subsection{最终标签得分}
对于每个标签类别,其最终得分为:

\[
Score_{tag} = \sum_{t \in keywords} Score_{position}(t) + \beta_{match} \times IsMatch_{title}(t)
\]

其中:
\begin{itemize}
\item $keywords$ 是标签类别的关键词集合
\item $\beta_{match} = 0.7$ 是关键词匹配权重
\item $IsMatch_{title}(t)$ 是一个指示函数,表示关键词是否在标题中匹配
\end{itemize}

\section{阈值和限制}

\subsection{标签匹配阈值}
文章只有在标签得分超过阈值时才会被分配该标签:

\[
Selected_{tag} = \begin{cases} 
1 & \text{if } Score_{tag} \geq \theta \\
0 & \text{otherwise}
\end{cases}
\]

其中 $\theta = 0.005$ 是标签匹配阈值。

\subsection{最大标签数限制}
每篇文章最多被分配 $k$ 个标签,其中 $k = 5$。

\end{CJK}
\end{document}